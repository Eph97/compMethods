% {{{
\documentclass[12pt]{article}
\usepackage[tmargin=0.75in,bmargin=0.75in,lmargin=0.9in,rmargin=0.9in]{geometry}

\usepackage{amsmath}
\include{latexsym}
\include{amssymb}
\usepackage{enumitem}
\usepackage[symbol, hang]{footmisc}
\usepackage{indentfirst}
\usepackage{amssymb}
\usepackage{graphicx,psfrag}

\def\F{\mathcal{F}}
\def\M{\mathcal{M}}
\def\dimspec{\mathfrak{D}}
\def\htop{h_{top}}
\def\trans{\mathcal{T}}
\def\G{\mathcal{G}}
\newcommand{\Z}{\mathbb{Z}}
\pagenumbering{gobble}
\def\O{\mathcal{O}}

\newcommand\NoIndent[1]{%
  \begingroup
  \par
  \parshape0
  #1\par
  \endgroup
}
% }}}

% Title {{{
\begin{document}

\begin{center}
{\large \bf ComputationalMethods }   \\ \large pset2 \\ Ephraim Sutherland
\end{center}
% }}}

\subsection*{Question 1}

\begin{enumerate}

\item 
p is [6.931894087109903, 6.931894087109903] and the iteration number is 40 \\ 
Now increasing demand by 10\% \\
p is [7.3173202293587245, 7.032580688033674] and the iteration number is 39

\item 
	Using Newton's method (with analytic jacobian) \\ 
x is [6.9315098138708215, 6.9315098138708215] and the iteration number is 37 \\ 
Now increasing demand by 10\% \\ 
x is [7.316899080901866, 7.032155448480992] and the iteration number is 16 \\


\end{enumerate} 

\subsection*{Question 2}

\begin{enumerate}
	\item  Solving the FOC, we can see that if we let $u(c) = \frac{1}{c}$ then we have
		 \begin{align*}
			 \underbrace{\max_{a_1} u(a_0 + y_0 - a_1) + \beta u((1+r)a_1 + y_1)}_\text{$ g(a_1) $} &= 0
		\end{align*}
		Thus we simply find the root of $g$. This then becomes
		\begin{align*}
			-u'(a_0 + y_0 - a_1) + (1+r)\beta u'((1+r)a_1 + y_1) &= 0 \\ 
			- \frac{1}{a_0 + y_0 - a_1} + \frac{\beta(1+r)}{(1+r)a_1 + y_1} &= 0 \\
		\end{align*}
		Solving for $a_1$ becomes
		\begin{align*}
			\frac{\beta(1+r)}{(1+r)a_1 + y_1} &= \frac{1}{a_0 + y_0 - a_1} \\
			\beta(1+r)a_0 + \beta(1+r)y_0 - \beta(1+r)a_1  &= (1+r)a_1 + y_1  \\
			\beta(1+r)a_0 + \beta(1+r)y_0 - y_1 &= (1+r)a_1 +\beta(1+r)a_1 \\
			a_1 &= \frac{\beta(1+r)a_0 + \beta(1+r)y_0 - y_1}{(1+ \beta)(1+r)}
		\end{align*}

	\item Recall that we were able to reduce our value function to
		$$ \underbrace{\max_{a_1} u(a_0 + y_0 - a_1) + \beta u((1+r)a_1 + y_1)}_\text{$ g(a_1) $} = 0$$ 
		using our optimal $a_1$ we can then reduce this to a function of $a_0$. Let $v$ be the value function, then we have

		$$v(a_0) = u(a_0 + y_0 -  \frac{\beta(1+r)a_0 + \beta(1+r)y_0 - y_1}{(1+ \beta)(1+r)}) + \beta u((1+r)\frac{\beta(1+r)a_0 + \beta(1+r)y_0 - y_1}{(1+ \beta)(1+r)}  +y_1)$$

	\item Using Newton's method.
with
$\beta =0.9$
$a_0=5$
$y_0 =20$
$y_1 =5$
$r=0.1$

optimal $a_1$ is 9.44976076555024 and the iteration number is 5

We also computed an Analytic answer of 9.44976076555024. This is very close to the answer computed using newton's method. In fact it's actually exactly the same which I find surprising (I promise I computed it analytically though).


\item Using Brent's method. The root (optimal $a_1$) is 9.452845792729457

	
\end{enumerate}

\subsection*{Question 3}
\begin{enumerate}
	\item FOC: 
		$$-u'(a_0 + y_0 - a_1) + \pi (1+r)u'((1+r)a_1 + y^{H}) + (1-\pi)(1+r)u'((1+r)a_1 + y^{L}) = 0$$
		using our function $u(c) = log(c)$ gives us 
		$$\frac{-1}{a_0 + y_0 - a_1} + \frac{\pi (1+r)}{(1+r)a_1 + y^{H}} + \frac{(1+\pi)(1+r)}{(1+r)a_1 + y^{L}} = 0$$

	\item solving for $a_1$:
\\		Using Newton's method.
\\$a_1$ is 10.18783705339048 and the iteration number is 6
\\Using Brent's method.
\\$a_1$ is 10.187912584223925
\\ We can clearly see that 10.1879 is larger than 9.4498 from question 2. Thus individuals with our given felicity function save more when there is uncertainty.

\item Without uncertainty:
\\$a_0$: 0, optimal $a_1$ is 7.0813397128751, with value: 5.159216446967338
\\$a_0$: 1, optimal $a_1$ is 7.555023923444977, with value: 5.235652357326654
\\$a_0$: 2, optimal $a_1$ is 8.028708133971293, with value: 5.309264750132845
\\$a_0$: 3, optimal $a_1$ is 8.502392344497608, with value: 5.380254521806624
\\$a_0$: 4, optimal $a_1$ is 8.976076555023923, with value: 5.448801892664775
\\$a_0$: 5, optimal $a_1$ is 9.44976076555024, with value: 5.515069146297081
\\$a_0$: 6, optimal $a_1$ is 9.923444976076556, with value: 5.579202930432368
\\$a_0$: 7, optimal $a_1$ is 10.397129186602873, with value: 5.641336200797893
\\$a_0$: 8, optimal $a_1$ is 10.870813397129188, with value: 5.7015898723587615
\\$a_0$: 9, optimal $a_1$ is 11.344497607655503, with value: 5.760074229189525
\\$a_0$: 10, optimal $a_1$ is 11.818181818181817, with value: 5.816890134066869

With uncertainty:
\\$a_0$: 0, optimal $a_1$ is 7.951495482645849, with value: 3.996386001921955
\\$a_0$: 1, optimal $a_1$ is 8.395335317574846, with value: 4.051246888021469
\\$a_0$: 2, optimal $a_1$ is 8.84104120272266, with value: 4.104039432261872
\\$a_0$: 3, optimal $a_1$ is 9.288453187251053, with value: 4.154922166236992
\\$a_0$: 4, optimal $a_1$ is 9.737427993787554, with value: 4.2040354365051895
\\$a_0$: 5, optimal $a_1$ is 10.18783705339048, with value: 4.2515041310105
\\$a_0$: 6, optimal $a_1$ is 10.63956478685629, with value: 4.297439910854835
\\$a_0$: 7, optimal $a_1$ is 11.092507101967277, with value: 4.341943050919057
\\$a_0$: 8, optimal $a_1$ is 11.546570079364056, with value: 4.385103968638063
\\$a_0$: 9, optimal $a_1$ is 12.001668822787405, with value: 4.427004502221634
\\$a_0$: 10, optimal $a_1$ is 12.45772645233553, with value: 4.467718986081811

Thus we can see that the value is less for every $a_0$ when we introduce uncertainty.


\end{enumerate}
\end{document}


